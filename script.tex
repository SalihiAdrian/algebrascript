\documentclass[a4paper, 10pt]{scrartcl}
\usepackage[utf8]{inputenc}
\usepackage[ngerman]{babel}
\usepackage[T1]{fontenc}
\usepackage{amsmath}
\usepackage{amssymb}
\usepackage{enumitem}
\title {Algebra und Funktionentheorie \\[1ex] \large}
\author{Prof. Riddle}
\date{31.Oktober.1981}

\begin{document}
\maketitle
\tableofcontents
\section{Grundlagen der Körpertheorie}
\textbf{Körper} (Definition)
Ein Tripel (K, +, $\cdot$), wobei K eine Menge und $+: K \times K \rightarrow K $ und $\cdot: K \times K \rightarrow K$ Verknüpfungen auf K sind, heißt Körper, wenn folgende Bedingungen erfüllt sind:
\begin{itemize}
\item $(K,+)$ ist eine abelsche Gruppe
\item $(K, \cdot$) ist eine abelsche Gruppe
\item Es gilt das Distributivgesetz: $a \cdot (b+c) = ab + ac$
\end{itemize}
Wir werden in Zukunft fast immer K statt $(K, +, \cdot)$ schreiben, weil das schneller geht und praktischer ist.
\\

\subsection{Körpererweiterungen}
In der folgenden Vorlesung werden wir uns intensiv mit Körpererweiterungen, d.h. Körpern als Teilmengen von anderen Körpern befassen. Wir werden diese Struktur, genannt "Körpererweiterung"\ als eigenes Studienobjekt und Körpererweiterungen mit weiteren Eigenschaften betrachten und mit dem Studium der Gruppentheorie und Polynomen in Beziehung setzen.
\\

\textbf{Teilkörper} (Definition)
Sei E ein Körper. Eine \textbf{Teilmenge} K von E heißt Teilkörper von E, falls K mit den von E induzierten Verknüpfungen selbst wieder ein Körper ist.

\textbf{Nachweis von Teilkörpereigenschaft} (Aussage): \\
Sei E ein Körper und K eine Teilmenge von E. Dann sind äquivalent:
\begin{itemize}
\item K ist Teilkörper von E
\item Es gilt: \begin{enumerate}[label=\roman*)]
\item $a-b \in K \text{ } \forall a \text{, }b \in K$
\item $ab^{-1} \in K \text{ } \forall a \in K \text{ and } b \in K^{x}$
\item $1 \in K$
\end{enumerate}
\end{itemize}

\textbf{Körpererweiterung} (Definition)\\
Sei E ein Körper und K ein Teilkörper von E. Dann nennen wir E eine Körpererweiterung von K und schreiben für diese Struktur: $E/K$.
\\
\\
\textbf{Anmerkung}\\
Im folgenden werden wir für Körpererweiterungen immer $E/K$ schreiben und für das Komplement zweier Mengen A und B: $A \backslash B$, d.h. für Körpererweiterungen '\ backward slash'\ und für Komplemente '˙forward slash'\ .
\\
\\
\textbf{Teilgebiete dieses Kapitels}\\
Das Thema "Körpererweiterungen"\ in diesem Kapitel werden wir in drei Unterkapitel sortieren. Wir werden sehen, dass:

\begin{itemize}

\item Körpererweiterungen immer eine Vektorraumstruktur tragen
\item Körpererweiterungen in einem Körper generiert werden können,. z.B. durch das Kompositum
\item Wir mittels Körperautomorphismen eine Brücke zu Polynomen schlagen können
\end{itemize}

\subsubsection{Körpererweiterungen generieren}
\textbf{Zwischenkörper} (Definition)\\
Sei $E/K$ eine Körpererweiterung. Ein Teilkörper L von E heißt \textbf{Zwischenkörper} der Körpererweiterung $E/K$, falls gilt: $E \supseteq L \supseteq K$.
\\
\\
\textbf{Schnitte von Teilkörpern sind Teilkörper} (Aussage)\\
Sei $(L_i)_{i \in I}$ eine Familie von Teilkörpern eines Körpers E. Dann ist:
$$\bigcap_{i \in I} L_i$$
auch ein Teilkörper von E.
\newline
\newline 
\textbf{Kompositum und andere} (Definition)
Sei E ein Körper und L, F Teilkörper von E. Dann
\begin{enumerate}[label=\roman*)]
\item heißt $LF := \bigcap \{\text{Teilkörper von E, die L und F enthalten} \}$ das Kompositum von L und F in E
\item Sei $E/K$ Körpererweiterung, $b_1,..,b_n \in K$ so heißt $$K(b_1,...,b_n):= \bigcap \{ \text{Teilkörper von E, die K und } b_1,..,b_n \text{  enthalten } \}$$
der von $b_1,...,b_n$ über $K$ erzeugte Teilkörper von $E$

\end{enumerate}
\textbf{Kompositum mit größerem Körper} (Ausage)\newline
Sei $E/K$ eine Körpererweiterung, $a_1,...,a_n \in E$ und $L$ Zwischenkörper von $E/K$. Dann gilt: $(K(a_1,..,a_n))L = L(a_1,...,a_n)$.\newline
\textbf{Beweis:}\newline
Dies ist leicht durch Anwendung der Definition beider Seiten der Gleichung zu zeigen.\newline
\newline
\textbf{Endlich erzeugt} (Definition) \newline
Gilt in der obigen Definition im Fall ii), dass: $K(b_1,...,b_n) = E$, so nennen wir die Körpererweiterung $E/K$ endlich erzeugt.



\subsubsection{Körpererweiterungen implizieren Vektorraumstruktur}
\textbf{Vektorraumstruktur für Körpererweiterungen} (Aussage)\\
Sei $E/K$ eine Körpererweiterung, dann ist E ein K-Vektorraum.
\textbf{Beweis}\\
Die Vektorraumaxiome sind leicht durch die Körpereigenschaften nachzuweisen und wird dem Leser überlassen. Hehehehe.
\\
\\
\textbf{Grad einer Körpererweiterung} (Definition)\\
Sei $E/K$ eine Körpererweiterung, dann ist der \textbf{Grad der Körpererweiterung $E/K$} gegeben durch $$[E:K]:=dim_K E$$
Also der Dimension des K-Vektorraumes E.\newline

\textbf{Gradformel} (Aussage)\\
Sei $E/K$ eine Körpererweiterung und $L$ ein Zwischenkörper dieser. Dann gilt die Formel:
$$[E:K] = [E:L] \cdot [L:K]$$

\textbf{Beweis der Gradformel}\\
Sei $(x_i)_{i \in I}$ eine L-Basis von E und $(y_j)_{j \in J}$ eine K-Basis von L. Dann ist $(x_i,y_j)_{i \in I, j \in J}$ eine K-Basis von E. Denn: \newline

\textbf{Erzeugendensystem}\\
Sei $e \in E$ ein beliebiges Element. Da $(x_i)$ eine L-Basis von E ist, existieren $\lambda_i \in L$ sodass $$e = \sum \lambda_i x_i$$
Da $(y_j)$ eine K-Basis von L ist, existieren $\mu_{ij} \in K$, sodass $\lambda_i = \sum_{j \in J} \mu_{ij} y_j$ für alle $i \in I$ und damit gilt:
$$ e = \sum \mu_{ij} x_i y_j$$
\newline

\textbf{Lineare Unabhängigkeit}\\
Seien nun $\lambda_{ij} \in K$, sodass $$\sum_{i \in I, j \in J} \lambda_{ij} y_j x_i = 0$$
Da $(x_i)$ L-Basis von E ist und $(\lambda_{ij}y_i)$ Skalare in L sind, muss gelten:
$$\lambda_{ij} y_j = 0 \text{ } \forall i \in I \text{ und } \forall j \in J$$
nach der Basiseigenschaft.\newline
Damit muss gelten $$\sum_{j \in J} \lambda_{ij}y_j = 0 \text{ } \forall i \in I$$ Da $(y_j)$ eine K-Basis von L ist, muss dann gelten: $$\lambda_{ij} = 0 \text{ } \forall i \in I \text{, } j \in J$$

Damit sind $(x_i y_j)$ linear unabhängige Vektoren in E.

Aus der Erzeugendeneigenschaft und der linearen Unabhängigkeit folgt die Behauptung.
\newline

\textbf{Endliche Körpererweiterung} (Definition)\\
Sei $E/K$ eine Körpererweiterung und $[E:K] < \infty$, dann nennen wir $E/K$ endlich.
\newline
\newline
\textbf{Endliche Körpererweiterungen sind endlich erzeugt} (Aussage)
Sei $E/K$ eine endliche Körpererweiterung. Dann ist sie auch endlich erzeugt.\newline
\textbf{Beweis}\newline
Sei $(b_1,...,b_n)$ eine K-Basis von E mit $n \in \mathbb{N}$. Sei $e \in E$.\newline
Es existieren $\lambda_i \in K$ mit $$e = \sum \lambda_i b_i$$ für $i \in \{1,...,n\}$
Da $\lambda_i \in K(b_1,..,b_n)$ und $b_1,...,b_n \in K(b_1,..,b_n)$, ist damit $e \in K(b_1,..,b_n)$, da $K(b_1,..,b_n)$ nach Definition ein Körper ist (und damit endliche Summen und Produkte von Elementen desselben immer in diesem enthalten sind).\newline
\newline
\newline
\textbf{E/K endl. g.d.w E/L und L/K endl.} (Aussage) \newline
Klar mit Gradformel

\subsubsection{Brücke zwischen Körperhomomorphismen und Polynomen}
\textbf{Körperhomomorphismus, (Körper-)Automorphismus}\newline
Seien $K$ und $K'$ Körper. Eine Abbildung $\sigma: K \rightarrow K'$ heißt Körperhomomorphismus, wenn folgende Bedingungen erfüllt sind:
\begin{itemize}
\item $\sigma (1) = 1$
\item $\sigma (a+b) = \sigma (a) + \sigma (b)$
\item $\sigma (ab) = \sigma (a) \sigma (b)$
\end{itemize}
Ein (Körper-)Automorphismus ist ein Körperhomomorphismus $\sigma K \rightarrow K'$, der bijektiv ist und für den gilt: $K = K'$.

\paragraph{Untergliederung des Themas Körperhomomorphismen}
Wir werden Körperhomomorphismen in zwei Teile gliedern. Der erste Teil befasst sich mit allgemeinen Eigenschaften von Körperhomomorphismen, dessen Wissen oft nützlich ist. Der zweite Teil befasst sich mit Automorphismen, deren Gruppenstruktur und Relevanz für Polynome. Hier finden wir die erste Brücke zwischen Automorphismen und Polynomen über deren Nullstellen.
\newline
\textbf{Allgemeine Eigenschaften von Körperhomomorphismen}\newline \newline 
\textbf{Alle Körperhomomorphismen sind injektiv} (Aussage)\newline
Sei $\sigma : K \rightarrow K'$ ein Körperhomomorphismus, dann ist $\sigma$ injektiv.\newline \newline
\textbf{Beweis:}\newline
Da $\sigma$ ein Körperhomomorphismus ist, ist es auch ein Gruppenhomomorphismus zwischen den abelschen Gruppen $(K,+)$ und $(K',+)$. Angenommen es existiert $x \in ker(\sigma) \backslash \{0\} $, dann gilt in $K'$: $$1 = \sigma(1) = \sigma (xx^{-1}) = \sigma(x) \sigma(x^{-1}) = 0 \sigma(x^{-1}) = 0$$
Demnach $0=1$. Da $K'$ ein Körper ist, muss $0 \neq 1$ gelten und wir haben einen Widerspruch.\newline
Demnach ist $ker(\sigma)=\{0\}$ und nach dem \textit{Monomorphiekriterium} ist $\sigma $ injektiv. \newline \newline


\textbf{Einbettung} (Definition) \newline
Ein Körperhomomorphismus wird aufgrund der obigen Aussage über dessen Injektivität auch Einbettung genannt.
\newline
\newline
\textbf{Einfache Aussagen zu Körperhomomorphismen} (Aussagen)\\
Seien $\sigma : K \rightarrow K'$ und $\mu : K \rightarrow L$ Körperhomomorphismen, dann ist wahr:
\begin{enumerate}[label=\roman*)]
\item $\{a \in K : \sigma (a) = \mu (a) \}$ ist Teilkörper von K
\item $\sigma(K)$ ist Teilkörper von $K'$
\end{enumerate}
\textbf{Beweis}: \\
Zu i): Der Beweis geht einfach mit dem Charakterisierungslemma zu Teilkörpern (ganz am Anfang des Kapitels 1)\\
Zu ii) Charakterisierungslemma zu Teilkörpern\\
Diese Beweise werden dem motivierten Leser (der du sicher bist) als Übungsaufgabe überlassen.\\\\
\textbf{Körperhomomorphismen und Kompositum} (Aussage) \\
Seien $L$ und $F$ Teilkörper eines Körpers $E$ und $K$ ein Körper. Sei $\sigma : E \rightarrow	K$ eine Einbettung. Dann gilt:
\begin{enumerate}[label=\roman*)]
\item $\sigma(LF) = \sigma(L) \sigma(F)$
\item Seien $a_1,..,a_n \in E$, dann: $\sigma(L(a_1,...,a_n)) = \sigma(L) (\sigma(a_1),..,\sigma(a_n))$
\item $\sigma(L \cap F) = \sigma(L) \cap \sigma(F)$
\end{enumerate}
\textbf{Beweis:}\\
Zu i): Wir verwenden die Kompositumseigenschaft.\\
($\supseteq$): Wegen $L \subseteq LF$ und $F \subseteq LF$ gilt: $$\sigma (L) \subseteq \sigma(LF) \text{, sowie } \sigma(F) \subseteq \sigma(LF)$$ damit gilt nach der Definition des Kompositums als kleinster Teilkörper von K, der $\sigma(L) \text{ und } \sigma(F)$ enthält:
$$\sigma(L) \sigma(F) \subseteq \sigma(LF)$$
($\subseteq$): Da gilt: $$L \text{, } F \subseteq \sigma^{-1} (\sigma(L) \sigma(F))$$ und $\sigma^{-1} (\sigma(L) \sigma(F))$ ein Teilkörper von $E$ ist. (Da $\sigma$ injektiv und somit $\sigma^{-1}$ als Einbettung von $\sigma(E) \rightarrow E$ gesehen werden kann.)\newline
Damit ist nach Definition des Kompositums von $L$ und $F$: $$\sigma^{-1} (\sigma(L) \sigma(F)) \supset LF$$
Damit folgt durch nochmalige Anwendung von $\sigma$ und dessen Injektivität: $$\sigma(L) \sigma(F) \supset \sigma(LF)$$
insgesamt gilt also die Gleichheit in i).\newline
\newline
Zu ii): Der Beweis folgt analog zu i), indem wir F durch $a_1, ...,a_n$ ersetzen und die Schritte passend durchgehen. \newline
\newline
Zu iii): Dies folgt aus der Injektivität von $\sigma$ und ist leicht nachzuprüfen, siehe Ana 1 Kurs.


\textbf{Automorphismen und deren Gruppenstruktur}\\
Wir erinnern uns an die Definition eines Körperautomorphismus. Auf diesen ist eine Gruppenstruktur mit der Komposition von Abbildungen $\circ$ definiert.\\
\textbf{Aut(K) bilden eine Gruppe} (Aussage)\newline
Sei $K$ ein Körper, dann is $(Aut(K), \circ)$ eine Gruppe.\\
\textbf{Beweis:}\\
Betrachte die Gruppe aller Abbildungen von K nach K mit der Komposition. Dann wende das Untergruppenkriterium an.
\newline
\newline
\textbf{Aut(E/K)} (Definition) \newline
Sei $E/K$ eine Körpererweiterung, dann ist: $$Aut(E/K) := \{\sigma : \sigma \text{ist Automorphismus von E mit } \sigma_{|K}=id_K \}$$

\textbf{Aut(E/K) ist Gruppe} (Aussage)\newline
$Aut(E/K)$ ist eine Untergruppe von $Aut(E)$
\newline
\textbf{Beweis:}\newline
Untergruppenkrtierium
\newline
\newline
\textbf{Brückensatz Automorphismen zu Nullstellen} (Aussage) \newline
Seien $E/K$ sowie $F/K$ Körpererweiterungen. Sei $\sigma : E \rightarrow F$ eine Einbettung und $f = \sum a_i X^i \in K[X]$ ein Polynom mit Nullstelle $\beta \in E$ und $deg(f)= n \in \mathbb{N}$.\newline
Dann ist $\sigma(\beta)$ eine Nullstelle von $f$ in $F$.\newline
\textbf{Beweis:}\\
Es gilt: $$f(\sigma(\beta)) = \sum a_i \sigma(\beta)^{i} = \sum a_i \sigma(\beta^{i}) = \sum \sigma(a_i \beta^{i}) =\sigma(\sum a_i \beta^{i}) = \sigma(0) = 0$$
\newline
\newline
\textbf{Anmerkung}\newline
Wir werden diesen Satz nun auf die Situation: $E=K(\beta)$ für eine Körpererweiterung $E/K$ anwenden ($\beta$ hier Nullstelle von $f \in K[X]$). Hier werden wir sehen, dass wir eine injektive Abbildung der $\sigma \in Aut(E/K)$  auf die Menge der Nullstellen von f in E finden können. So können wir also mit der Nullstelle $\beta$ und den Elementen in $Aut(E/K)$ neue Nullstellen generieren, falls $\#Aut(E/K) > 1$ ist. Die Idee ist K mittels Adjunktion weiterer Nullstellen zu erweitern, sodass wir dann alle Nullstellen von f finden. (Glaube ich)
\newline
\newline
\textbf{Anwendung des obigen Satzes auf E=K($\beta$)} (Aussage) \newline
Sei $E/K$ eine Körpererweiterung, $f \in K[X]$ mit Nullstelle $\beta \in E$, sodass gilt: $E=K(\beta)$. \newline Dann ist die Abbildung $Aut(E/K) \rightarrow \{\text{Nullstellen von f in E} \}$, wobei $\sigma \mapsto \sigma(\beta)$, eine Injektion.\newline
\textbf{Beweis:}\newline
Seien $\sigma \text{, } \tau \in Aut(E/K)$ mit $\sigma (\beta) = \tau (\beta)$. Aus Obigem wissen wir: $$M:=\{a \in E : \sigma (a) = \tau (a) \}$$ ist ein Teilkörper von $E$. Wegen $\beta \in M$ und $K \subset M$ gilt $M \supseteq K(\beta)$. Also damit $$M \supseteq K(\beta)=E$$ also $M=E$.\newline
Damit ist $\sigma = \tau$.
\newline
\newline
\textbf{Anmerkung:} (Aussage)\newline
Damit gilt natürlich auch $\#Aut(E/K) \leq \{\text{Nullstellen von f in E} \}$
\paragraph{Können wir immer eine Körpererweiterung von K finden, in der eine Nullstelle von f existiert?}
Oben haben wir erwähnt, dass wir (auf eindeutige Weise) weitere Nullstellen von $f \in K[X]$ finden können, indem wir Automorphismen einer Körpererweiterung auf diese Nullstelle anwenden. Woher wissen wir aber, dass es auch immer eine solche Körpererweiterung gibt, in der wir eine Nullstelle für f finden? Wenn das nicht gilt, wäre unser obiges Resultat zwar schön, aber vielleicht nicht für alle Situationen hilfreich (wenn z.B. gar keine Körpererweiterung mit einer Nullstelle von f existiert).\newline
Unsere Sorge ist unbegründet, wir können immer eine Körpererweiterung von K finden, sodass $f$ in dieser eine Nullstelle besitzt (und damit immer obigen Satz anwenden, um nach neuen Nullstellen zu suchen). Um dies zu beweisen, werden wir die Ringtheorie lernen und entwickeln.
\par

\end{document}



